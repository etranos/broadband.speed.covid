\documentclass[Royal,times,sageh]{sagej}

\usepackage{moreverb,url,natbib, multirow, tabularx}
\usepackage[colorlinks,bookmarksopen,bookmarksnumbered,citecolor=red,urlcolor=red]{hyperref}



\usepackage{hyperref}
\usepackage[utf8]{inputenc}
\usepackage{dcolumn}
\def\tightlist{}
\newcommand{\beginappendix}{\setcounter{table}{0} \renewcommand{\thetable}{A\arabic{table}} \setcounter{figure}{0} \renewcommand{\thefigure}{A\arabic{figure}}}


\begin{document}

\title{Working from home and digital divides: resilience during the pandemic}

\runninghead{}

\author{}

\affiliation{}



\begin{abstract}
This paper offers a new perspective on telecommuting from the viewpoint
of the complex web of digital divides. Using the UK as a case study,
this paper studies how the quality and reliability of internet services,
as reflected in \emph{experienced} internet upload speeds during the
spring 2020 lockdown, may reinforce or redress the spatial and social
dimensions of digital divisions. Fast, reliable internet connections are
necessary for the population to be able to work from home. Although not
every place hosts individuals in occupations which allow for
telecommuting nor with the necessary skills to effectively use the
internet to telecommute, good internet connectivity is also essential to
local economic resilience in a period like the current pandemic.
Employing data on individual broadband speed tests and state-of-the-art
time-series clustering methods, we create clusters of UK local
authorities with similar temporal signatures of experienced upload
speeds. We then associate these clusters of local authorities with their
socioeconomic and geographic characteristics to explore how they overlap
with or diverge from the existing economic and digital geography of the
UK. Our analysis enables us to better understand how the spatial and
social distribution of both occupations and online accessibility
intersect to enable or hinder the practice of telecommuting at a time of
extreme demand.
\end{abstract}

\keywords{covid; internet; working from home; broadband speed; time-series
clusters;}

\maketitle

\hypertarget{sec:1}{%
\section{Introduction}\label{sec:1}}

During the pandemic, working from home using digital technologies,
whether partially or exclusively, was transformed from a niche means of
accessing work, albeit one that had been on a slow, upward trend, to a
widespread way of life in many countries. The ability to work from home
or telecommute meant millions retained their jobs and, to a varying
extent, maintained productivity during periods of strict lockdowns
around the world. However, this ability has not been evenly distributed
socially or spatially, creating new intersections of economic and
digital divisions. On one side are those who can work from home,
supported by digital technologies, and have thus been able to enjoy both
economic resilience and greater personal safety. On the other side,
previously employed individuals have been forced to accept furlough or
redundancy packages unless they are part of the cadre of essential
workers, who are potentially at high risk of infection. Whilst the basis
for this pandemic-generated divide has been viewed as mainly
occupational, here we consider whether it is also technological and,
consequently, geographical.

Using the UK as a case study, this paper aims to understand how the
quality and reliability of internet service, as reflected in
\emph{experienced} internet speeds during the spring 2020 lockdown, may
reinforce or redress the spatial and social dimensions of digital
divisions. We employ volunteered geographic data on individual broadband
speed tests and state-of-the-art time-series clustering methods to
create clusters of UK local authorities with similar temporal signatures
of experienced internet speeds. We then associate these clusters of
local authorities with their socioeconomic and geographic
characteristics to explore how they overlap with or diverge from the
existing economic and digital geography of the UK. Our analysis enables
us to better understand how the spatial and social distribution of
education, occupation and online accessibility intersect to enable or
hinder the practice of telecommuting at a time of extreme demand. We
also consider what lessons can be learned from this time for a future
where telecommuting is likely to remain a more common means of accessing
work, at least in comparison to the pre-Covid era.

The capability to work from home has previously been studied from the
perspective of whether work tasks in a given occupation both can be and
are allowed to be performed using digital technologies independently of
location or co-location with colleagues, including supervisors
\citep{allen2015effective, singh2013modeling}. However, successful
telecommuting also requires that the quality and reliability of digital
services, particularly home internet connection speeds, enable the
completion of work tasks with a minimum of delay or interruption. Prior
to the pandemic, the performance of broadband services with respect to
telecommuters was never tested at scale, as working from home and
connecting to colleagues and workplace resources via the internet was
the purview of a small minority of workers. Instead, leisure use in the
evening, when video streaming services are at their peak, has been used
to benchmark broadband performance and service delivery by different
Internet Service Providers (ISPs) \citep{ofcom2017}.

Yet the shift towards telecommuting during various stages of lockdown
around the world has been drastic and there are speculations that
post-Covid, the tendency to work from home will be much higher, raising
questions around whether internet services can accommodate the increased
demand. For example, \(47\)\% of people in employment in the UK worked
solely from home in April \(2020\), whilst the same figure only reached
\(5\)\% the year before \citep{ons2020, ons2020lm2019}. A back of the
envelope calculation suggests that up to 40\% of the labour force could
work from home on an ongoing basis \citep{batty2020editorial}. Similar
figures have been reported for other countries
\citep{felstead2020homeworking}. Approximately \(37\)\% of the European
workforce worked from home in April \(2020\) with countries like Finland
reaching \(60\)\% \citep{eurofound2020}. In the US, almost half of the
working population worked from home during the same period because of
the pandemic \citep{brynjolfsson2020covid}, and a recent estimate
indicated that \(37\)\% of all jobs in the US can be permanently
performed entirely from home \citep{NBERw26948}. None of these changes
could have happened in the absence of reliable information and
communication technology (ICT) infrastructure -- both in terms of
software and hardware. But while software innovations are easily
diffused across space and society\footnote{See for example the huge
  success of videoconferencing apps such as Zoom \citep{marks2020zoom}.},
the same does not apply for ICT hardware infrastructure such as internet
broadband connectivity.

The literature describes first level digital divides in terms of the
availability and quality of internet connectivity, such as that manifest
in different geographies in the UK
\citep{riddlesden2014broadband, philip2017digital}. Second level digital
divides consider the presence or lack of the necessary skills to
effectively utilise digital technologies and the internet
\citep{blank2014dimensions, van2011internet}. The third level focuses on
the heterogenous returns of internet usage among different socioeconomic
groups and, consequently, how digital technologies can assist in
bridging or further enhancing existing socioeconomic divides.
\citep{stern2009levels, van2014digital, van2015third}. The capability to
telecommute is related to all three levels of digital divides, but more
importantly leads to differentiated outcomes regarding the economic
resilience of people and places to overcome a systemic shock such as the
current pandemic.

As the quality of internet infrastructure and services, as well as the
concentration of different occupations are spatially dependent and
clustered in space, our approach offers a framework for understanding
the impact of and interactions between digital divisions geographically
and socioeconomically. By asking how resilient broadband speeds, and
particularly upload speeds, are as experienced in different parts of the
UK during a time of extreme demand, we interrogate which places benefit
from the greater economic resilience digital technologies can offer. The
structure of this paper is as follows. First we review the literature on
telecommuting and digital divides to better understand their structural
and spatial development pre-pandemic, and thus their importance to the
economic resilience of different places. We then describe our data and
methodology. Our results section first offers a classification of how
internet services vary across clusters of UK local authorities and then
assesses whether these clusters replicate or repudiate other
socio-economic and geographic patterns of economic resilience. We
conclude with a discussion of the insights we have gained from our new
perspective on digital divisions.

\hypertarget{sec:2}{%
\section{Literature review}\label{sec:2}}

\hypertarget{sec:2.1}{%
\subsection{From telecommuting to \#WFH}\label{sec:2.1}}

In this analysis, the terms `telecommuting' and `working from
home\footnote{See also the popular social media hashtag \#WFM}' are used
interchangeably, as most remote labour during the Covid-19 crisis was
carried out in the homes of individual employees rather than any other
location \citep{eurofound2020}. However, previous research has explored
how telecommuting can occur in other places, including satellite offices
or on public transport \citep{felstead2012rapid, siha2006telecommuting}.
Previous research has also used a variety of definitions to measure the
level of telecommuting within different workforces, distinguishing
between those directly employed, indirectly employed, self-employed,
full-time or part-time, and those who use digital technologies to work
remotely full-days or part-days
\citep{allen2015effective, bailey2002review, haddad2009examination}. No
matter the definition, the option and capability to telecommute or work
from home has never been equally distributed spatially or
socio-economically any more than different industries and employment
opportunities have. Studies from the United States, the Netherlands, and
the UK found that telecommuters are most likely to hold professional,
managerial, and technical occupations where the workforce is better
educated and wealthier, and that there is suppressed demand among women
and part-time workers
\citep{headicar2016move, peters2004employees, singh2013modeling}.

Opportunities for working from home during the current pandemic have
likewise not been equally spread across the workforce.
\citet{NBERw26948} indicated that in the US, managers, educators, those
working in computer-related occupations, finance, and law can easily
work from home, and that occupations with opportunities to telecommute
are associated with higher earnings. This is not the case for the
workforce occupied in more spatially fixed occupations, from farming,
construction and manufacturing to hospitality and care services. In the
US, these occupations tend to be lower-income, non-white, without a
university degree, live in rental accommodation and lack health
insurance \citep{NBERw27085}. Similar trends can be observed for other
countries. For example, \(75\)\% of workers with tertiary education
worked from home in Europe during spring \(2020\), whilst only \(34\)\%
of workers with secondary education and \(14\)\% of those primary
education did so \citep{eurofound2020}.

\hypertarget{sec:2.2}{%
\subsection{Digital divides and economic resilience}\label{sec:2.2}}

Our understanding of telecommuting as a product of enabled occupations
can be described as a manifestation of the third level digital divide,
as those who are able to use digital technologies to work from home
benefit from a high rate of return on their use of the internet in terms
of autonomy, flexibility, and time saved from commuting
\citep{peters2004employees, siha2006telecommuting, singh2013modeling}.
These returns have been even greater during the Covid-19 crisis, when
those with the capability to telecommute also have the ability to
maintain their employment whilst protecting their health. However, the
success of these arrangements has been dependent upon the first level
digital divide, which is associated with access to and quality of
internet connectivity. \citet{SALEMINK2017360} provides a systematic
review of the pre-pandemic, first level digital divide in infrastructure
quality between urban and rural areas in various advanced economies.
Rural areas, predictably, fare worse, yet as \citet{blank2018local}
highlight, variation in individual internet uptake and use is a product
of more complex spatial and demographic characteristics than simple
rurality or urbanisation. Yet whether this variation in infrastructure
quality affects the spatial footprint of telecommuting has not
previously been investigated, in part because telecommuting has not
previously been a cause of concentrated demand and pressure on internet
services.

There are indications that those who purchase high speed connections
consume more data of all sorts and use their connections for a greater
variety of purposes \citep{hauge2011consumer}. There is also a
correlation between access to internet services and a reduction in
household transport spend \citep{bris2017ict}. Independently of whether
the additional internet use and reduced travel is because of increased
telecommuting, these studies suggest that better internet services
enable households to make savings and efficiencies, an example of the
first level digital divide reinforcing the third level.

Multi-layered digital divides may also intersect with material divides
and the economic geography of the UK. The regional economic resilience
literature underlines the differentiated capacity of cities and regions
to escape or recover from economic crises
\citep{martin2012regional, kitsos2018economic}. As different places have
different industrial and occupational profiles, these affect their
aggregated potential capacity for telecommuting. Such profiles are
associated with longstanding inequalities in the UK and their spatial
representation as a North-South divide \citep{martin_north_south}.
Various studies have illustrated severe inequalities between the north
and the south regions of England in terms of skills and human capital,
unemployment, productivity and prosperity
\citep{lee2014grim, mccann2020perceptions, dorling2018peak}. Some
scholars have even argued that the UK suffers from some of the highest
levels of interregional inequalities in the global north
\citep{gal2018reducing, mccann2016uk}. All three levels of digital
divides are associated with or shaped by the geography of the UK. Yet
this is the first time that the intersection of digital and material
divides is relevant to understanding the economic resilience of places
and large swathes of the population, as digital technologies became an
essential tool of productivity during the Covid-19 pandemic.

The extreme demand during the pandemic thus provides a new opportunity
to understand how internet infrastructure quality, and reliability
affects telecommuting, particularly in light of the high volumes of
bandwidth-intensive video conferencing required in order to avoid the
face-to-face contact that could increase the spread of infection. We
seek to answer how internet service resilience could contributes to or
reduce economic resilience when the latter is dependent upon the
capability to work from home. We also aim to improve our understanding
of the impact of first level digital divisions on telecommuting, and
whether this results in more fundamental third level digital divisions
than has previously been perceived.

\hypertarget{sec:3}{%
\section{Methods and data}\label{sec:3}}

\hypertarget{sec:3.1}{%
\subsection{Time-Series clustering}\label{sec:3.1}}

Our chosen methodological framework is cluster analysis, which can be
defined within machine learning approaches as an unsupervised learning
task, partitioning unlabelled observations into homogeneous groups known
as clusters \citep{montero2014tsclust}. The key idea is that
observations within clusters tend to be more similar than observations
between clusters. Clustering is particularly useful for exploratory
studies as it identifies structures within the data
\citep{aghabozorgi2015time}. Cluster analysis is a widely used in
geography \citep{gordon1977classification, everitt1974cluster}, for
example to solve \emph{regionalisation} problems
\citep{niesterowicz2016}. Clustering methods are also the basis of
\emph{geodemographics}, a research domain which aims to create small
area indicators or typologies of neighbourhoods based on diverse
variables \citep{SINGLETON2009289, harris2005geodemographics}. These
studies usually employ cross-sectional data, and most clustering
problems in geography deal with observations that are fixed in time.
However, for this paper we are interested in internet speeds, which vary
over time. Therefore, we create clusters of local authorities in the UK
with similar temporal signatures of experienced internet speeds.

To do so, we employ time-series clustering methods, which have been
developed to deal with clustering problems linked to, for instance,
stock or other financial data, economic, governmental or medical data as
well as machine monitoring
\citep{aggarwal2013time, aggarwal2001surprising, hyndman2015large, WARRENLIAO20051857}.
The main challenge, which does not apply to cross-sectional clustering
problems, is data dimensionality, with a multiplicity of data points for
every individual object included in the data set, and how their value
changes dynamically as a function of time \citep{aghabozorgi2015time}.
This high dimensionality leads to (i) computational and algorithmic
challenges regarding handling these data and building algorithms to
perform clustering over long time-series, and (ii) open questions
regarding the choice of similarity measures in order to cluster similar
times-series objects together considering the whole length of the
time-series and overcoming issues around noise, outliers and shifts
\citep{lin2004iterative, aghabozorgi2015time}.

For this paper we utilise a category of time-series clustering methods
known as shape-based approaches. These methods match two separate
time-series objects based on the similarity of their shapes through the
calculation of distances between the shapes, and are better equipped to
capture similarities between short length time-series
\citep{aghabozorgi2015time}, such as our data. We thus identify clusters
of UK local authorities with similar temporal signatures -- i.e.~shapes
-- of experienced internet speeds. The clusters are identified using the
common partitioning algorithm, where no clusters overlap, known as
\emph{k}-means. This iterative algorithm is popular because of the
simplicity of the implementation and the interpretability of the
results. It begins with defining the desired number of clusters:
\emph{k}. Then each observation is randomly assigned to a cluster from
the \([1,k]\) space. This initial cluster assignment is followed by
iterations in order to minimise the distance between the centroids of
the clusters and the observations assigned to these clusters
\citep{james2013introduction}.

There are a number of differences between the application of
\emph{k}-means for cross-sectional and times-series data. Instead of
creating clusters around centroids, a common approach is to create
clusters around \emph{medoids}, which are representative time-series
objects with a minimal distance to all other cluster objects
\citep{sardatime}. Also, instead of calculating the Euclidean distance
between centroids and data points, more complex distance measures need
to be employed to capture the similarity between a time-series object
and a medoid. A common distance measure for shape-based time-series
clustering is Dynamic Time Warping (DTW), an algorithm comparing two
time-series objects to find the optimum warping path between them. DTW
is widely used in order to overcome limitations linked to the use of
Euclidean distance
\citep{sardatime, berndt1994using, ratanamahatana2004everything}. The
\texttt{R} package \texttt{dtwclust} has been used for the time-series
clustering \citep{dtwclust}.

\hypertarget{sec:3.2}{%
\subsection{Experienced Broadband Speeds}\label{sec:3.2}}

To assess the internet quality and reliability across local authorities
in the UK, we utilise unique data comprising individual internet speed
tests from Speedchecker Ltd\footnote{\url{https://www.broadbandspeedchecker.co.uk/}}.
This is a private company that allows internet users to check their own
broadband upload and download speeds, and stores every speed-check with
timestamp and geolocation information. These data have been used before
to assess digital divides \citep{riddlesden2014broadband} and the impact
of local loop unbundling regulatory processes
\citep{nardotto2015unbundling}, and we followed the former's approach to
remove outliers. By using this volunteered geographic data, we are able
to assess the internet speed \emph{experienced} by users, which may
differ from the maximum speeds \emph{advertised} by ISPs. We are
particularly interested in upload speeds and the frequency of speed
tests over the period from March to May \(2020\), as government
statements indicate this is when UK workers were first told to work from
home if at all possible \citep{GovUK2020}. Average upload speeds are
slower than average download speeds, at \(9.3\)Mb/s mean upload speed
for the whole sample, compared to \(29.6\)Mb/s for download speeds, but
they are also less associated with internet-based, high-demand, leisure
activities such as video streaming. Therefore, upload speeds are more
relevant to work-related activities such as uploading documents or
two-way audio, video, and text-based communication systems.

The frequency of speed tests was important in identifying the temporal
profile which would give us most insight into experienced internet
service and resilience over units of time. Whilst there is an overall
trend of increased testing from March to April and then a slight
reduction from April to May, this trend masks substantial variation by
not only the day of the week, but also time of day, as can be seen in
Figure \ref{test2020}. Thus, a daily aggregation of upload speeds would
mask the variation in experienced service over the course of each
weekday. Furthermore, the importance of this variation is highlighted by
a comparison with the same period in \(2019\), as in Figure
\ref{test2019}, when the volume of testing and thus of experienced
internet service quality peaked in the evening, presumably in response
to demand for leisure activities and download speeds. In contrast, the
majority of the increase in testing in \(2020\) is during the working
day, creating a new morning peak in Figure \ref{test2020}. Therefore, we
include a measure of hourly variation in our temporal profiles to
reflect the change in users' perception of the workday reliability of
internet services.

However, there were insufficient observations -- only \(631\) speed
tests per LAD on average -- for each for each working hour of each
working day in each Local Authority District (LAD) to profile speeds at
that level of detail. Spatial aggregation was also necessary because we
could not follow individuals or households and connect data points.
Therefore, we aggregate the \(241,088\) individual, geolocated and
time-stamped speed-checks during the \(13\) weeks of March to May
inclusive for weekdays in \(2020\) by each hour of the day and day of
the week. As our research aims to identify the geography of internet
service resilience for work purposes, bank holidays and the hours
between midnight and \(6:00\) were excluded, as well as weekend days.
The composite week time-series thus comprise \(18\) hours multiplied by
\(5\) weekdays or \(90\) time points per series. The time-series were
calculated for each of the \(382\) LADs in the UK, standardised, and
then a \emph{k}-means partitioning around medoids clustering algorithm
was applied using DTW. We initially run the algorithm for
\(k \in \mathbb{N} \bigcap [5,15]\) and used cluster validity indices
(CVIs) to pick the optimal solution of \(k = 13\). Following
\citet{sardatime} the majority vote for the following CVIs was used:
Silhouette (max), Score function (max), Calinski-Harabasz (max),
Davies-Bouldin (min), Modified Davies-Bouldin (DB*, min), Dunn (max),
COP (min).

\begin{figure}
\centering
\includegraphics[width=0.85\textwidth,height=0.5\textheight]{figures/time.var.plot2020.png}
\caption{Speed tests over time, 2020 \label{test2020}}
\end{figure}

\begin{figure}
\centering
\includegraphics[width=0.85\textwidth,height=0.5\textheight]{figures/time.var.plot2019.png}
\caption{Speed tests over time, 2019 \label{test2019}}
\end{figure}

In Section \protect\hyperlink{sec:4.1}{4.1}, we review the temporal
profile of upload speed by hour of the day and day of the composite week
for each cluster. Since the quality and reliability of internet services
vary in time and space due to both supply and demand-side influences, we
also use a number of different measures to describe experienced upload
speeds per cluster. These include: i) mean, experienced connection
speed, ii) standard deviation or the amount of fluctuation from the
mean, and iii) the variation in speeds during the new morning peak of
testing when working from home is more likely to take place. We take
account of all three measurements in order to determine how resilient
broadband speeds are as experienced in different parts of the UK during
a time of extreme demand.

The cause of these different experiences of broadband resilience may
vary between and within clusters, as they may reflect either patterns of
demand or quality of infrastructure. Our approach is also limited by
potential endogeneity, as for example, better quality connections with
high mean speeds may enable more working from home, but greater demand
can cause slower speeds, less reliability, or greater variability of
speed at different times of day or week. Therefore, we avoid attributing
any cause to our analysis of the experienced level of quality and
reliability of upload speeds. Instead, we run an auxiliary regression to
understand how the spatial and temporal patterns of internet service
relate to the economic geography of the UK. More specifically, we
estimate the following multinomial logit model:

\begin{align}
Pr(Y_{i}=j) = \frac{exp^{X_{i}\beta_{j}}}{\sum_{i=1}^j exp^{X_{i}\beta_{j}}}
\begin{cases}
    i = 1, 2, ... , N \\  
    j = 1, 2, ... , J
\end{cases}\label{eq1}
\end{align}

Based on the outcomes of the time-series clustering, we identify \(J\)
distinct and crisp clusters. We then regress this cluster membership
against a vector \(X_{i}\) of socio-economic and geographic variables,
which are discussed in detail in the relevant Section
\protect\hyperlink{sec:4.2}{4.2}. Because we cannot identify individuals
or households and consequently aggregated our data at the LAD level, our
results offer correlations between the socioeconomic characteristics of
certain geographic locations and internet service quality, not a record
of who was telecommuting. Such individual data could be found though
surveys, but these offer less detailed information about the experience
of internet resilience due to enforced demand, which is the main
contribution of this paper. Our auxiliary regression, therefore,
provides an indication of how internet connectivity can reinforce or
redress existing spatial and social inequalities in different places.
However, it opens a path to future research by highlighting the
importance of understanding of how telecommuting capabilities and
digital infrastructure divisions intersect.

\hypertarget{sec:4}{%
\section{Results}\label{sec:4}}

\hypertarget{sec:4.1}{%
\subsection{Upload Clusters / cluster description}\label{sec:4.1}}

The temporal profiles of the local authority clusters have been
summarised in Figures \ref{UpClusterL} and \ref{UpClusterS} and Table
\ref{up.cluster.descr}. The graphs show a composite profile of mean
upload speeds per hour per day for each cluster, with the largest five
clusters, in terms of the LAD membership and population, in Figure
\ref{UpClusterL}, and the next six in Figure \ref{UpClusterS}. These
figures and table provide a comprehensive overview of the quality and
reliability of experienced broadband in different parts of the UK.

\begin{figure}
\includegraphics[width=0.95\linewidth]{figures/upClusterL} \caption{\label{UpClusterL}Temporal profilies for upload speed large clusters}\label{fig:unnamed-chunk-2}
\end{figure}

\begin{figure}
\includegraphics[width=0.95\linewidth]{figures/upClusterS} \caption{\label{UpClusterS}Temporal profilies for upload speed small clusters}\label{fig:unnamed-chunk-3}
\end{figure}

\begin{table}[!htbp] \centering 
  \caption{Upload speed cluster characteristics\label{up.cluster.descr}} 
  \label{} 
\footnotesize 
\begin{tabular}{@{\extracolsep{0pt}} ccccccc} 
\\[-1.8ex]\hline 
\hline \\[-1.8ex] 
Cluster & N. of LADs & LAD population & mean speed & SD speed & mean AM speed & mean PM speed \\ 
\hline \\[-1.8ex] 
1 & 5 & 343100 & 8557 & 6139 & 7747 & 9563 \\ 
2 & 2 & 265600 & 10922 & 6687 & 9674 & 10645 \\ 
3 & 4 & 474700 & 10201 & 5658 & 9470 & 11236 \\ 
4 & 1 & 91100 & 9689 & 6122 & 7816 & 9689 \\ 
5 & 1 & 79800 & 10127 & 6024 & 9030 & 11101 \\ 
6 & 155 & 29535700 & 9397 & 5839 & 9161 & 9580 \\ 
7 & 4 & 559800 & 10119 & 6102 & 9813 & 11070 \\ 
8 & 5 & 436300 & 9429 & 6254 & 8682 & 10434 \\ 
9 & 32 & 6355500 & 10878 & 5957 & 10832 & 11071 \\ 
10 & 4 & 699600 & 10795 & 6005 & 9258 & 10697 \\ 
11 & 33 & 5771400 & 10845 & 5936 & 10781 & 10988 \\ 
12 & 10 & 1544900 & 9551 & 6166 & 9254 & 9048 \\ 
13 & 126 & 20277700 & 8392 & 5849 & 8299 & 8522 \\ 
\hline \\[-1.8ex] 
\multicolumn{7}{l}{Note: All speed measures are upload speeds} \\ 
\end{tabular} 
\end{table}

The second largest cluster, comprising \(126\) local authorities and
over \(20\) million people, is cluster \(13\), which has the slowest
aggregate mean upload speed, and the second highest ratio of the
standard deviation to the mean. This suggests that those living in local
authorities in this cluster experienced some of the lowest quality
broadband services in terms of upload speeds in the UK. However, Figure
\ref{UpClusterL} shows that the high standard deviation, which is one
indication of unreliable internet, did not disproportionately affect the
morning peak from \(9:00\)-\(10:59\), when upload speeds were, on
average, only \(2.6\)\% slower than in the evening peak period between
\(19:00\) and \(20:59\), when entertainment purposes are likely to be
using the most bandwidth. In comparison, the five LADs in cluster \(1\),
home to \(343\) thousand people, not only experience the second slowest
mean upload speeds and the highest ratio of standard deviation to the
mean, but are also much more affected during the morning peak.

Meanwhile, those living in the largest cluster -- \(6\), with \(155\)
LADs and \(29.5\) million people -- experienced aggregate mean upload
speeds of about \(1\)Mb/s faster than those in cluster \(13\), but still
lower than the other three large clusters and most of the smaller
clusters, suggesting a middling quality of service. The temporal profile
for cluster \(6\) in Figure \ref{UpClusterL} shows that upload speeds
are highest at \(6:00\) on a Monday morning and between \(23:00\) and
midnight on Wednesday and Thursday, but tend to be lower during the
working day. Furthermore, experienced mean upload speeds in the morning
peak are \(4.4\)\% lower than in the evening peak -- a greater, more
noticeable change than any of the other large clusters experience,
suggesting poorer reliability during the working day. This difference is
less, however than any of the clusters included in Figure
\ref{UpClusterS}.

Clusters \(8\) and \(12\) also have mean upload speeds under \(10\)Mb/s,
but higher than clusters \(1\), \(13\), and \(6\). However, mean upload
speeds are much lower between \(9:00-10:59\) than between
\(19:00-20:59\) in cluster \(8\), but slightly faster in the morning
than in the evening in cluster \(12\). Indeed, cluster \(12\) is the
only cluster to experience higher speeds in the morning peak compared to
the evening, and thus the only cluster where the temporal profile of
internet use is closer to what might have been expected pre-pandemic.
Among the other clusters, however, the reliability of internet services
during the working day still varies considerably. Interpreting this
variation from the large spikes and dips shown on Figures
\ref{UpClusterL} and \ref{UpClusterS} is difficult, but the statistics
in Table \ref{up.cluster.descr} show that clusters \(9\) and \(11\) have
the most reliable internet services. The ratio of standard deviation to
mean in both these clusters is lowest, and speeds are only about 2\%
slower in the morning than the evening. Mean speeds are also higher than
in any other cluster, excluding cluster \(2\), where measures of
reliability suggest poorer performance.

Thus, broadband services in clusters \(9\) and \(11\), home to over
twelve million people, performed the best during the study period, in
terms of both quality and reliability. In Figure \ref{UpClusterL},
cluster \(11\) shows more noticeable peaks and troughs, with the lowest
points occurring, on average, between \(6:00-7:00\) on Monday morning,
\(14:00-15:00\) on Friday, and \(16:00-17:00\) on Thursday, not the
workday peak. Furthermore, these slow periods offer better speeds than
the average hourly profile of cluster \(13\). Clusters \(3\), \(7\) and
\(10\) also have relatively high mean upload speeds. Clusters \(3\) and
\(10\) pair high mean speeds with low standard deviations relative to
the mean speeds, suggesting reliability and resilience, as well as
quality broadband services. Cluster \(7\) has a higher ratio of standard
deviation to mean, but there is less difference in average speeds
between the morning and evening peaks than in clusters \(3\) and \(10\).

In summary, LADs in clusters \(9\) and \(11\) experienced resilient
broadband internet that could support high levels of telecommuting.
Those in clusters \(2\), \(3\), \(7\), and \(10\) also experience higher
than average mean speeds and rank middle to high on measures of service
reliability. These LADs are \emph{not} on the wrong side of the first
level digital divide, but how likely are they to be able to take
advantage of their resilient ICT infrastructure and services? Meanwhile,
cluster \(6\) is not only the largest in terms of number of LADs and
population, it has the closest mean upload speed to the pre-clustered
average for the whole sample. As well as average quality internet
services, those in cluster \(6\) also experience average reliability for
work purposes, ranking fifth behind the four other clusters with
populations over one million, but ahead of the smaller clusters.
Clusters \(8\) and \(12\) are also close to average mean upload speeds,
but show very different patterns in terms of reliability, whilst
clusters \(1\) and \(13\) appear to suffer most from a lack of quality
internet services, with slow speeds and high standard deviations. With
those in cluster \(1\) more likely to experience that poor reliability
during the morning peak, is this first level digital divide occurring in
areas where few are occupationally able to telecommute anyway, and what
are the implications for economic resilience?

\hypertarget{sec:4.2}{%
\subsection{Post-clustering regression analysis}\label{sec:4.2}}

\begin{sidewaystable}[!htbp] \centering 
  \caption{Auxiliary multinomial regression of upload speed clusters on socio-economic and geographic LAD variables\label{aux}} 
  \label{} 
\tiny 
\begin{tabular}{@{\extracolsep{5pt}}lccccccccccc} 
\\[-1.8ex]\hline 
\hline \\[-1.8ex] 
\\[-1.8ex] & 1 & 2 & 3 & 6 & 7 & 8 & 9 & 10 & 11 & 12 & 13 \\ 
\\[-1.8ex] & (1) & (2) & (3) & (4) & (5) & (6) & (7) & (8) & (9) & (10) & (11)\\ 
\hline \\[-1.8ex] 
 pop, 2018 & $-$0.00004$^{***}$ & 0.00002$^{*}$ & 0.00001 & 0.00002$^{***}$ & 0.00002$^{***}$ & 0.00000 & 0.00002$^{***}$ & 0.00002$^{***}$ & 0.00002$^{***}$ & 0.00002$^{**}$ & 0.00002$^{***}$ \\ 
  & (0.00002) & (0.00001) & (0.00001) & (0.00001) & (0.00001) & (0.00002) & (0.00001) & (0.00001) & (0.00001) & (0.00001) & (0.00001) \\ 
  & & & & & & & & & & & \\ 
 job density, 2018 & $-$0.536$^{***}$ & $-$1.834$^{***}$ & $-$0.132$^{***}$ & $-$0.925$^{***}$ & $-$1.208$^{***}$ & $-$0.299$^{***}$ & $-$1.746$^{***}$ & $-$1.436$^{***}$ & 3.350$^{***}$ & 3.400$^{***}$ & 0.630$^{***}$ \\ 
  & (0.00000) & (0.00000) & (0.00000) & (0.00000) & (0.00000) & (0.00000) & (0.00000) & (0.00000) & (0.00000) & (0.00000) & (0.00000) \\ 
  & & & & & & & & & & & \\ 
 distance to nearest met. area & $-$0.034$^{***}$ & $-$0.014$^{***}$ & 0.002$^{***}$ & $-$0.020$^{***}$ & $-$0.074$^{***}$ & $-$0.044$^{***}$ & $-$0.013$^{***}$ & $-$0.036$^{***}$ & $-$0.031$^{***}$ & $-$0.036$^{***}$ & $-$0.024$^{***}$ \\ 
  & (0.0005) & (0.001) & (0.0002) & (0.002) & (0.0001) & (0.0002) & (0.002) & (0.0005) & (0.0003) & (0.0002) & (0.002) \\ 
  & & & & & & & & & & & \\ 
 distance to London & 0.007$^{***}$ & 0.002 & $-$0.016$^{***}$ & 0.001 & 0.004$^{***}$ & 0.004$^{***}$ & $-$0.002$^{*}$ & 0.005$^{**}$ & $-$0.002 & 0.003 & 0.006$^{***}$ \\ 
  & (0.001) & (0.002) & (0.0004) & (0.001) & (0.001) & (0.001) & (0.001) & (0.002) & (0.002) & (0.002) & (0.001) \\ 
  & & & & & & & & & & & \\ 
 South of the UK & $-$0.410$^{***}$ & $-$1.451$^{***}$ & $-$0.039$^{***}$ & $-$0.111$^{***}$ & $-$0.048$^{***}$ & $-$0.841$^{***}$ & $-$0.798$^{***}$ & 1.492$^{***}$ & 0.610$^{***}$ & 0.798$^{***}$ & 2.403$^{***}$ \\ 
  & (0.00000) & (0.00000) & (0.00000) & (0.00001) & (0.00000) & (0.00000) & (0.00001) & (0.00000) & (0.00001) & (0.00001) & (0.00001) \\ 
  & & & & & & & & & & & \\ 
 managerial jobs, 2020 & 0.939$^{***}$ & 0.704$^{***}$ & 0.435$^{***}$ & 0.704$^{***}$ & 0.316$^{***}$ & 0.786$^{***}$ & 0.576$^{***}$ & 0.311$^{***}$ & 0.476$^{***}$ & 0.594$^{***}$ & 0.615$^{***}$ \\ 
  & (0.00004) & (0.0001) & (0.00004) & (0.00004) & (0.00002) & (0.0001) & (0.00003) & (0.00003) & (0.00002) & (0.00004) & (0.00003) \\ 
  & & & & & & & & & & & \\ 
 tech jobs, 2020 & 0.096$^{***}$ & $-$0.257$^{***}$ & $-$0.071$^{***}$ & $-$0.111$^{***}$ & $-$0.206$^{***}$ & 0.199$^{***}$ & $-$0.126$^{***}$ & $-$0.606$^{***}$ & $-$0.180$^{***}$ & $-$0.398$^{***}$ & $-$0.112$^{***}$ \\ 
  & (0.00004) & (0.00004) & (0.00004) & (0.00003) & (0.00003) & (0.0001) & (0.00003) & (0.00003) & (0.00002) & (0.00004) & (0.00003) \\ 
  & & & & & & & & & & & \\ 
 skilled trade jobs, 2020 & 0.651$^{***}$ & 0.160$^{***}$ & $-$0.191$^{***}$ & 0.236$^{***}$ & 0.604$^{***}$ & $-$0.184$^{***}$ & 0.205$^{***}$ & 0.597$^{***}$ & 0.108$^{***}$ & $-$0.022$^{***}$ & 0.295$^{***}$ \\ 
  & (0.00004) & (0.00004) & (0.00003) & (0.00003) & (0.00003) & (0.00004) & (0.00002) & (0.00003) & (0.00003) & (0.00004) & (0.00003) \\ 
  & & & & & & & & & & & \\ 
 professional jobs, 2020 & $-$0.118$^{***}$ & $-$0.234$^{***}$ & $-$0.121$^{***}$ & $-$0.172$^{***}$ & $-$0.514$^{***}$ & $-$0.172$^{***}$ & $-$0.349$^{***}$ & $-$0.351$^{***}$ & $-$0.245$^{***}$ & $-$0.344$^{***}$ & $-$0.229$^{***}$ \\ 
  & (0.00005) & (0.0001) & (0.0001) & (0.00005) & (0.00003) & (0.0001) & (0.00004) & (0.0001) & (0.00003) & (0.00005) & (0.0001) \\ 
  & & & & & & & & & & & \\ 
 administrative jobs, 2020 & 0.019$^{***}$ & $-$0.836$^{***}$ & $-$0.040$^{***}$ & $-$0.117$^{***}$ & $-$0.139$^{***}$ & 0.206$^{***}$ & $-$0.058$^{***}$ & $-$0.200$^{***}$ & $-$0.055$^{***}$ & $-$0.168$^{***}$ & $-$0.177$^{***}$ \\ 
  & (0.00003) & (0.00002) & (0.00004) & (0.00001) & (0.00003) & (0.00003) & (0.00001) & (0.00002) & (0.00002) & (0.00002) & (0.00002) \\ 
  & & & & & & & & & & & \\ 
 leisure jobs, 2020 & $-$0.198$^{***}$ & $-$0.180$^{***}$ & $-$0.225$^{***}$ & $-$0.476$^{***}$ & $-$0.654$^{***}$ & $-$0.820$^{***}$ & $-$0.537$^{***}$ & $-$0.935$^{***}$ & $-$0.353$^{***}$ & $-$0.625$^{***}$ & $-$0.491$^{***}$ \\ 
  & (0.00002) & (0.00004) & (0.00004) & (0.00002) & (0.00002) & (0.00003) & (0.00002) & (0.00001) & (0.00002) & (0.00003) & (0.00002) \\ 
  & & & & & & & & & & & \\ 
 machine operation jobs, 2020 & $-$0.336$^{***}$ & 0.207$^{***}$ & 0.392$^{***}$ & 0.010$^{***}$ & $-$0.433$^{***}$ & 0.139$^{***}$ & $-$0.099$^{***}$ & $-$0.139$^{***}$ & $-$0.144$^{***}$ & 0.098$^{***}$ & $-$0.179$^{***}$ \\ 
  & (0.00002) & (0.00003) & (0.00003) & (0.00002) & (0.00001) & (0.00003) & (0.00002) & (0.00001) & (0.00001) & (0.00002) & (0.00001) \\ 
  & & & & & & & & & & & \\ 
 earnings, 2019 & $-$0.003$^{*}$ & 0.010$^{***}$ & 0.012$^{***}$ & 0.020$^{***}$ & 0.027$^{***}$ & 0.001 & 0.020$^{***}$ & 0.016$^{***}$ & 0.015$^{***}$ & 0.025$^{***}$ & 0.014$^{***}$ \\ 
  & (0.002) & (0.002) & (0.002) & (0.001) & (0.001) & (0.003) & (0.001) & (0.002) & (0.001) & (0.001) & (0.001) \\ 
  & & & & & & & & & & & \\ 
 n. business est. per hab., 2019 & 0.126$^{***}$ & $-$0.120$^{***}$ & $-$0.094$^{***}$ & $-$0.133$^{***}$ & 0.123$^{***}$ & $-$0.051$^{***}$ & $-$0.334$^{***}$ & $-$0.133$^{***}$ & $-$0.150$^{***}$ & 0.289$^{***}$ & 0.377$^{***}$ \\ 
  & (0.00000) & (0.00000) & (0.00000) & (0.00000) & (0.00000) & (0.00000) & (0.00000) & (0.00000) & (0.00000) & (0.00000) & (0.00000) \\ 
  & & & & & & & & & & & \\ 
 NVQ4+ & $-$0.141$^{***}$ & 0.064$^{***}$ & $-$0.091$^{***}$ & $-$0.070$^{***}$ & $-$0.010$^{***}$ & 0.004$^{***}$ & 0.016$^{***}$ & 0.170$^{***}$ & $-$0.110$^{***}$ & $-$0.038$^{***}$ & $-$0.035$^{***}$ \\ 
  & (0.0001) & (0.0001) & (0.0001) & (0.0001) & (0.0001) & (0.0002) & (0.0001) & (0.0001) & (0.0001) & (0.0001) & (0.0001) \\ 
  & & & & & & & & & & & \\ 
 AM tests per hab., 2020 & 0.0005$^{***}$ & $-$0.002$^{***}$ & $-$0.005$^{***}$ & 0.010$^{***}$ & 0.0004$^{***}$ & $-$0.001$^{***}$ & $-$0.002$^{***}$ & $-$0.005$^{***}$ & $-$0.013$^{***}$ & $-$0.001$^{***}$ & 0.016$^{***}$ \\ 
  & (0.000) & (0.000) & (0.000) & (0.000) & (0.000) & (0.000) & (0.000) & (0.000) & (0.000) & (0.000) & (0.000) \\ 
  & & & & & & & & & & & \\ 
 Virgin Media \%, 2020 & $-$0.044$^{***}$ & 1.578$^{***}$ & 1.210$^{***}$ & 0.248$^{***}$ & $-$1.724$^{***}$ & $-$0.242$^{***}$ & 3.109$^{***}$ & $-$0.085$^{***}$ & 1.214$^{***}$ & $-$3.889$^{***}$ & $-$0.745$^{***}$ \\ 
  & (0.00000) & (0.00000) & (0.00000) & (0.00000) & (0.00000) & (0.00000) & (0.00000) & (0.00000) & (0.00000) & (0.00000) & (0.00000) \\ 
  & & & & & & & & & & & \\ 
 Constant & 0.321$^{***}$ & $-$0.436$^{***}$ & 0.199$^{***}$ & $-$2.953$^{***}$ & 0.278$^{***}$ & 0.002$^{***}$ & $-$0.866$^{***}$ & 0.788$^{***}$ & 2.600$^{***}$ & 0.017$^{***}$ & 0.022$^{***}$ \\ 
  & (0.00000) & (0.00000) & (0.00000) & (0.00000) & (0.00000) & (0.00000) & (0.00000) & (0.00000) & (0.00000) & (0.00000) & (0.00000) \\ 
  & & & & & & & & & & & \\ 
\hline \\[-1.8ex] 
McFadden's R squared & 0.338 & 0.338 & 0.338 & 0.338 & 0.338 & 0.338 & 0.338 & 0.338 & 0.338 & 0.338 & 0.338 \\ 
N & 323 & 323 & 323 & 323 & 323 & 323 & 323 & 323 & 323 & 323 & 323 \\ 
Akaike Inf. Crit. & 1,148.027 & 1,148.027 & 1,148.027 & 1,148.027 & 1,148.027 & 1,148.027 & 1,148.027 & 1,148.027 & 1,148.027 & 1,148.027 & 1,148.027 \\ 
\hline 
\hline \\[-1.8ex] 
\textit{Note:}  & \multicolumn{11}{r}{$^{*}$p$<$0.1; $^{**}$p$<$0.05; $^{***}$p$<$0.01} \\ 
\end{tabular} 
\end{sidewaystable}

Using an auxiliary multinomial logit regression, we test whether the
clusters that have higher mean speeds and more reliable services consist
of LADs that are more urban, affluent, and / or more likely to benefit
from a choice of high quality internet services. We also estimate which
of our clusters are more likely to have a higher proportion of
occupations where the nature of the work enables telecommuting. The
results of the auxiliary regression are presented in Table \ref{aux}.
The dependent variable is the LAD cluster membership as described in the
\protect\hyperlink{sec:3}{methods and data} section and equation
\ref{eq1}. Each column represents a different cluster. The reference
case is cluster \(4\), which includes only the local authority of
Hambleton in North Yorkshire, a rural area of just over ninety thousand
people. Mean, experienced upload speeds in cluster \(4\) (see Table
\ref{up.cluster.descr}) are between the average speeds for the \(13\)
clusters (\(9.9\)Mb/s) and the pre-clustered average for the whole
sample (\(9.3\)Mb/s). However, the standard deviation for cluster \(4\)
and the difference between average speeds in the morning compared to the
evening peak periods are indications of worse reliability than many of
the other clusters. Hence, the results in Table \ref{aux} should be seen
as relative rather than absolute probabilities.

First, we control for the number of speed tests run per cluster
inhabitant between \(9:00\)-\(10:59\) as well as the share of fast
Virgin Media internet connections. Regarding the former, we expect
people in LADs with more unreliable connections to test their internet
speeds more often, and the results validate our priors. Meanwhile, fast
Virgin Media cable connections have historically only been available to
\(45\)\% of premises in the UK \citep{ofcom2016}, where the more
lucrative and competitive market originally attracted the cable TV
provider. Those in clusters \(2\), \(3\), \(9\) and \(11\) benefit from
a higher proportion of Virgin connections, which is an indication that
people in these clusters are more likely to live in urban areas, with
more choice of broadband services. In other words, they are more likely
to be on the right side of the first level, infrastructure-based digital
divide, as we expected from the analysis in Section
\protect\hyperlink{sec:4.1}{4.1}.

We employ distance from London and from the nearest metropolitan area
(including London) as two variables depicting peripherality, urban
structure and, potentially, first level digital divides. The broadband
speed tests run in the authorities in cluster \(3\) are more likely to
be taking place close to London than those run in any of the other
clusters, and two of the four authorities in cluster \(3\) are the
London commuter towns of Harlow and Luton. However, even though London
was also included in the variable calculating distance from the centre
of one of either the ten largest metropolitan areas in England, or
Glasgow or Cardiff, tests run in cluster \(3\) are likely to be furthest
away. Thus, it is important to consider the membership of each cluster
as well as the regression results. Corby and Eastbourne, the other two
authorities in cluster \(3\) are large, accessible towns, although not
part of metropolitan areas.

\begin{figure}
\includegraphics[width=0.95\linewidth]{figures/map.up.clusters} \caption{\label{map.up.clusters}Upload speed clusters for LADs}\label{fig:unnamed-chunk-6}
\end{figure}

Meanwhile, LADs in cluster \(7\) are most likely to be near the centre
of a large metropolitan area, even though the four local authorities of
cluster \(7\) include no central urban boroughs and only one LAD that is
part of a metropolitan area of governance - Tameside in Greater
Manchester. This may explain why those in cluster \(7\) are the second
least likely to be served by Virgin Media. It is also a demonstration of
the complexity of both experienced broadband upload speeds as captured
by time-series clustering, and the geography of first level digital
divides as a product of quality as well as availability. In comparison,
cluster \(1\), which our analysis suggests lacks broadband resilience,
contains five, mainly rural authorities, but they are closer to a
metropolitan area than authorities in cluster \(3\), although furthest
from London, perhaps because they are scattered around the country --
see Figure \ref{map.up.clusters}.

Internet resilience is also more nuanced than our -- arguably crude --
dummy variable depicting the North-South economic divide, which assigns
\(1\) to LADs located in Greater London, Southeast, Southwest and East
of England regions, and \(0\) to the rest. The authorities most likely
to be in the South are those in cluster \(13\), which was identified as
having the slowest mean upload speeds of any of the clusters, and thus a
low level of service. However, cluster \(13\) includes some rural,
remote areas of the country, such as Northwest Scotland, Cornwall and
Powys in Wales as shown on Figure \ref{map.up.clusters}. It also
includes major metropolitan centres in the North -- Liverpool, Newcastle
-- and South -- Bristol, and nine (of \(32\)) London Boroughs. There are
also plenty of Southern home county and suburban areas. The standard
deviation measure for cluster \(13\) is high, but speed variation is low
during the morning peak, suggesting that the estimates for reliability
are inconsistent. Considering that this is one of the largest clusters,
and thus the averages incorporate more noise than some of the smaller
clusters, it may be that the LADs in this cluster do not all suffer
equally from a first level digital divide.

Yet we need to consider the results for other variables in Table
\ref{aux} to better determine whether clusters \(1\) and \(13\), which
appear to suffer most from poor quality internet services, are also more
likely to have a low skilled workforce, less able to benefit from
telecommuting. Cluster \(1\) has the lowest proportion of educated
people, and the lowest earnings, despite recording the highest
proportion of managerial and professional jobs, and the second highest
proportion of tech jobs. Rural areas such as those in cluster \(1\) are
home to many older, retired people \citep{blank2018local}, which might
explain these results or perhaps, as these figures are relative, we
could note that cluster \(1\) also has a greater proportion of skilled
trades than other clusters. In any case, it appears that the first level
digital divide reinforces other inequities in cluster \(1\), where the
even slower than average morning upload speeds shown in Table
\ref{up.cluster.descr} suggest that internet users were more active
during the working day. Meanwhile, those in cluster \(13\) are more
likely to earn more and have a better education despite poor internet
services. Cluster \(13\) also has the most businesses per inhabitant,
but is somewhere in the middle in terms of job density. If this is an
indication of a high number of SMEs, it could explain the variable
internet quality, considering that small businesses have not been seen
as the most valuable customers for higher speed broadband services if
they are not located near residential customers \citep{ofcom2016}.

Cluster \(6\) is the largest cluster, and thus, like cluster \(13\),
there is more noise within the averages we use to measure internet
quality and reliability. Our results indicate average mean speeds, and
Table \ref{aux} shows that cluster \(6\) also falls towards the middle
of the clusters on many of the socioeconomic variables. It ranks third
or fourth out of the eleven in terms of the likelihood of having a
higher proportion of managerial, professional, and tech jobs, as well as
higher earnings, but is fourth from bottom for educational attainment.
The LADs in cluster \(6\) are also diverse, with few truly remote areas,
but urban areas throughout England, including Birmingham, Leeds,
Sheffield, twelve London Boroughs, and many suburban areas and smaller
cities like Oxford and Cambridge. The capital cities of the other UK
nations, Belfast, Cardiff and Edinburgh, are also in this cluster,
suggesting perhaps that the lower level of reliability discussed in
Section \protect\hyperlink{sec:4.1}{4.1} may be due to increased demand,
e.g.~for telecommuting, despite lacking the most resilient internet
connections.

In contrast, Clusters \(9\) and \(11\) enjoy resilient internet
connections, but LADs in Cluster \(9\) are more likely to host highly
educated individuals with higher earnings than cluster \(11\), which has
a negative coefficient for the NVQ4 variable. Cluster \(11\) has fewer
businesses per inhabitant and the second highest job density and cluster
\(9\) the second lowest. These coefficients might indicate that
resilient broadband infrastructure generates higher returns for those in
cluster \(9\), where slightly more slowdown in the AM peak was detected
-- see Table \ref{up.cluster.descr}. LADs in cluster \(9\) are less
likely to be in the South and those in cluster \(11\) more likely,
although the coefficient for cluster \(9\) might be skewed by the
presence of the Scottish cities of Glasgow and Aberdeen. Scotland has a
different economic profile than England. Still, both clusters consist of
larger LADs in terms of population, including districts within five of
England's ten largest cities, and a number of other stand-alone urban
areas and large market towns -- see Appendix
\protect\hyperlink{appendix1}{1}. These urban locations are on the right
side of the first level digital divide, and the regression results
suggest that internet resilience is supporting a wide range of small and
large urban economies.

LADs in clusters \(2\), \(3\), \(7\), \(10\) are also on the right side
of the first level digital divide, experiencing higher than average mean
upload speeds and ranking high to middle on measures of service
reliability. LADs in clusters \(2\) and \(10\) are also more likely to
have highly skilled workers to take advantage of working from home
opportunities. Cluster \(2\) is comprised of just two LADs, with the
lowest job density of any cluster. Cluster \(10\) is comprised of four
LADs, including two peripheral suburban areas of Birmingham, the East
London Borough of Newham, and Dundee. Suburbs are considered the most
likely urban form in which telecommuters live \citep{e2018does}, and
Dundee has a reputation for tech startups \citep{technation2017}. LADs
in cluster \(7\), which include the suburbs South of Belfast, a suburban
district of Greater Manchester, and some other Midlands and North
Yorkshire towns and villages have the highest earnings, and thus may
also be making the most of their resilient internet. In contrast, LADs
in cluster \(3\) are associated more with lower skills and have the
highest proportion of individuals in machine operation jobs. Arguably,
LADs in clusters \(3\) benefit the least from their resilient internet
infrastructure in an era when working from home became a vehicle for
economic resilience.

Finally, clusters \(8\) and \(12\) consist of LADs enjoying close to
average mean upload speeds, but opposing patterns of internet
reliability during the morning peak. LADs in cluster \(8\) are likely to
host more individuals employed in tech and administrative occupations
than any other cluster, as well as many in managerial occupations,
whilst the opposite applies to skilled trade and leisure occupations.
These LADs are characterised by a very small positive likelihood of more
educated residents, but are not significantly likely to earn more than
other clusters. Including suburbs near Leicester, Cardiff, and
Birmingham and south of London, the LADs in cluster \(8\) seem to likely
to have the skills and occupations that would benefit from quality
internet services, but suffer from poor internet resilience and
reliability.

In comparison, LADs in cluster \(12\) are less likely to be able to
benefit from quality internet services if they had them, with fewer
individuals achieving NVQ4 or better and lower levels of occupations
that would benefit from homeworking. Yet LADs in this cluster benefit
from the second highest level of earnings, and have the largest stock of
businesses per inhabitant and the highest job density. This density of
businesses could be why cluster \(12\) is the only cluster with higher
speeds in the morning peak during the study period. If people are at
home, business premises might well have been abandoned. Cluster \(12\)
is home to \(1.5\) million people spread across twelve LADs, from London
to Wales as listed in Appendix \protect\hyperlink{appendix1}{1}. This
spatial diversity demonstrates that the temporal clustering of internet
resilience is not necessarily spatially dependent, and digital divides
do not necessarily overlap with economic ones.

\hypertarget{sec:5}{%
\section{Conclusions}\label{sec:5}}

This paper offers a new perspective on telecommuting from the viewpoint
of the complex web of digital divides. We employ novel data regarding
experienced upload speeds and time-series clustering methods, a family
of unsupervised machine learning techniques which are rarely utilised in
geographical research. Fast, reliable internet connections are necessary
for the population to be able to work from home. Although not every
place hosts individuals in occupations which allow for telecommuting nor
with the necessary skills to effectively use the internet to
telecommute, this paper raises the issue that places without good
internet connectivity will still struggle more than other places to
achieve economic resilience in a period like the current pandemic when
internet resilience is so vital. Indeed, our analysis demonstrated that
the temporal profiles of twelve of our thirteen clusters had slower
upload speeds in the morning than in the evening. The opposite is likely
to have been the norm prior to the pandemic, as level of demand and
bandwidth management is the most common cause of temporal variation in
experienced speeds, and why evening download speeds, rather than daytime
upload speeds, have been used to benchmark the performance of internet
services. Thus, the new patterns can be taken as evidence of widespread
telecommuting and other daytime internet use which changed the temporal
profile of internet activity throughout the UK, not just in areas with
more digital industry or better skills.

Upload speeds have not previously been seen as integral to universal
service, considering there has never before been such extreme demand for
telecommuting and operations such as video calls. This may be why the
average upload speeds for the largest two clusters -- \(6\) and \(13\)
-- contain so many LADs that are centres of the knowledge economy, from
Oxford and Cambridge in the former, to areas like Reading in the latter,
which has the highest concentration of digital businesses in the country
\citep{technation2017}. Thus, whilst some areas suffer from an
intersection of digital and economic divides, such as those in cluster
\(1\), other areas, including cluster \(8\) as well as \(6\) and \(13\)
also found their digital infrastructure to be less than reliable when
confronted with the sudden change in the timing and type of demand. In
contrast, LADs such as the digital hub, Milton Keynes, in cluster \(9\)
were able to benefit both from reliable internet connections and
populations which were familiar with working from home and could
capitalise on their digital infrastructure. Yet the nuanced picture we
gained through our analysis of the UK case study suggests that quality
internet connectivity may also have enabled other LADs, such as those in
cluster \(3\) or \(11\) to gain ground during the pandemic despite lower
skill levels.

Digital infrastructure which considers upload speeds and working day
reliability as well as availability are likely to be particularly
important in a future where telecommuting might be a more common means
of accessing work and broadband services must be fit for purpose,
although the long-term effects of such drastic changes in telecommuting
and attitudes towards working from home are difficult to predict.
Nevertheless, they span various aspects of economy and society: from
changes to transportation planning due to altered commuting patterns to
changes in land use and urban planning to accommodate people who work
from home \citep{BUDNITZ2020102713, ELLDER2020102777}, and from
productivity and innovation changes to changes in agglomeration
externalities and the attraction of large cities \citep{econobs}.
Further research may be able to measure the economic resilience of the
different clusters of places discussed in this paper once this pandemic
is firmly past. However, our analysis demonstrates that the economic
resilience made possible by working from home cannot be understood
without considering the underpinning digital divides and cannot be
achieved without planning for how the levels of digital, social and
economic divides might intersect.

\hypertarget{appendix1}{%
\section{Appendix 1}\label{appendix1}}

This is the LAD cluster membership for the upload speed timeseries.

\textbf{Cluster 1: } Ceredigion, Darlington, Eden, Rossendale, Rutland

\textbf{Cluster 2: } East Lothian, North East Lincolnshire

\textbf{Cluster 3: } Corby, Eastbourne, Harlow, Luton

\textbf{Cluster 4: } Hambleton

\textbf{Cluster 5: } Fylde

\textbf{Cluster 6: } Allerdale, Amber Valley, Angus, Ashfield, Ashford,
Aylesbury Vale, Barnet, Basingstoke and Deane, Bath and North East
Somerset, Belfast, Bexley, Birmingham, Blaenau Gwent, Bournemouth,
Christchurch and Poole, Bradford, Braintree, Brentwood, Bridgend,
Bromley, Broxtowe, Bury, Calderdale, Cambridge, Canterbury, Cardiff,
Castle Point, Chelmsford, Cheltenham, Cherwell, Chesterfield, City of
Edinburgh, Clackmannanshire, Colchester, Copeland, County Durham,
Coventry, Croydon, Dartford, Daventry, Denbighshire, Derby, Derry City
and Strabane, Dorset, Ealing, East Ayrshire, East Hampshire, East
Lindsey, East Northamptonshire, East Renfrewshire, East Riding of
Yorkshire, East Suffolk, Eastleigh, Elmbridge, Enfield, Falkirk,
Fareham, Gateshead, Gedling, Gosport, Gravesham, Great Yarmouth,
Guildford, Harborough, Haringey, Harrogate, Harrow, Hart, Hartlepool,
Havering, Herefordshire, County of, High Peak, Hinckley and Bosworth,
Horsham, Islington, Kettering, King's Lynn and West Norfolk, Kingston
upon Thames, Kirklees, Leeds, Leicester, Lincoln, Maidstone, Maldon,
Mansfield, Medway, Mendip, Mid Sussex, Middlesbrough, Monmouthshire,
Moray, New Forest, Newcastle-under-Lyme, Newport, North Ayrshire, North
East Derbyshire, North Hertfordshire, North Kesteven, North Lanarkshire,
North Lincolnshire, North Norfolk, North Tyneside, North West
Leicestershire, Northumberland, Nuneaton and Bedworth, Oxford,
Pembrokeshire, Pendle, Renfrewshire, Ribble Valley, Rochford, Runnymede,
Rushcliffe, Ryedale, Salford, Sefton, Sheffield, Shropshire, Solihull,
South Ayrshire, South Hams, South Holland, South Lanarkshire, South
Oxfordshire, South Staffordshire, St Albans, Staffordshire Moorlands,
Stockport, Stoke-on-Trent, Surrey Heath, Sutton, Swale, Tamworth,
Tendring, Test Valley, Thurrock, Tonbridge and Malling, Torfaen,
Wakefield, Warrington, Warwick, Wealden, Wellingborough, West Berkshire,
West Dunbartonshire, West Lancashire, West Lothian, West Oxfordshire,
West Suffolk, Wigan, Wiltshire, Woking, Worcester, Wrexham, Wycombe,
York

\textbf{Cluster 7: } Lisburn and Castlereagh, Selby, Tameside, Wyre
Forest

\textbf{Cluster 8: } Lichfield, Oadby and Wigston, Tandridge, Vale of
Glamorgan, West Devon

\textbf{Cluster 9: } Aberdeen City, Barnsley, Broxbourne, Charnwood,
Chorley, Erewash, Glasgow City, Greenwich, Halton, Havant, Knowsley,
Lewisham, Merton, Mid and East Antrim, Milton Keynes, Newark and
Sherwood, Northampton, Oldham, Portsmouth, Richmond upon Thames, Rugby,
Sandwell, South Derbyshire, South Kesteven, South Northamptonshire,
Southampton, Spelthorne, Stockton-on-Tees, Telford and Wrekin, Trafford,
Walsall, Welwyn Hatfield

\textbf{Cluster 10: } Bromsgrove, Cannock Chase, Dundee City, Newham

\textbf{Cluster 11: } Barking and Dagenham, Blaby, Blackpool, Bolsover,
Brent, Burnley, Caerphilly, Carlisle, Causeway Coast and Glens, Crawley,
Doncaster, Dudley, Hertsmere, Hounslow, Hyndburn, Ipswich, Isles of
Scilly, Kensington and Chelsea, Lewes, Manchester, North Warwickshire,
Norwich, Nottingham, Peterborough, Redditch, Rochdale, Scarborough,
Slough, St.~Helens, Stevenage, Sunderland, Vale of White Horse,
Wolverhampton

\textbf{Cluster 12: } Ards and North Down, Conwy, East Staffordshire,
Epping Forest, Fenland, Hammersmith and Fulham, Preston, Rhondda Cynon
Taf, Three Rivers, Westminster

\textbf{Cluster 13: } Aberdeenshire, Adur, Antrim and Newtownabbey,
Argyll and Bute, Armagh City, Banbridge and Craigavon, Arun, Babergh,
Barrow-in-Furness, Basildon, Bassetlaw, Bedford, Blackburn with Darwen,
Bolton, Boston, Bracknell Forest, Breckland, Brighton and Hove, Bristol,
City of, Broadland, Camden, Carmarthenshire, Central Bedfordshire,
Cheshire East, Cheshire West and Chester, Chichester, Chiltern, City of
London, Cornwall, Cotswold, Craven, Dacorum, Derbyshire Dales, Dover,
Dumfries and Galloway, East Cambridgeshire, East Devon, East
Dunbartonshire, East Hertfordshire, Epsom and Ewell, Exeter, Fermanagh
and Omagh, Fife, Flintshire, Folkestone and Hythe, Forest of Dean,
Gloucester, Gwynedd, Hackney, Hastings, Highland, Hillingdon,
Huntingdonshire, Inverclyde, Isle of Anglesey, Isle of Wight, Kingston
upon Hull, City of, Lambeth, Lancaster, Liverpool, Malvern Hills,
Melton, Merthyr Tydfil, Mid Devon, Mid Suffolk, Mid Ulster, Midlothian,
Mole Valley, Na h-Eileanan Siar, Neath Port Talbot, Newcastle upon Tyne,
Newry, Mourne and Down, North Devon, North Somerset, Orkney Islands,
Perth and Kinross, Plymouth, Powys, Reading, Redbridge, Redcar and
Cleveland, Reigate and Banstead, Richmondshire, Rother, Rotherham,
Rushmoor, Scottish Borders, Sedgemoor, Sevenoaks, Shetland Islands,
Somerset West and Taunton, South Bucks, South Cambridgeshire, South
Gloucestershire, South Lakeland, South Norfolk, South Ribble, South
Somerset, South Tyneside, Southend-on-Sea, Southwark, Stafford,
Stirling, Stratford-on-Avon, Stroud, Swansea, Swindon, Teignbridge,
Tewkesbury, Thanet, Torbay, Torridge, Tower Hamlets, Tunbridge Wells,
Uttlesford, Waltham Forest, Wandsworth, Watford, Waverley, West Lindsey,
Winchester, Windsor and Maidenhead, Wirral, Wokingham, Worthing,
Wychavon, Wyre

\hypertarget{appendix2}{%
\section{Appendix 2}\label{appendix2}}

\begin{table}[!htbp] \centering 
  \caption{Descriptive statistics for the auxiliary regression explanatory variables\label{descr.aux}} 
  \label{} 
\footnotesize 
\begin{tabular}{@{\extracolsep{0pt}}lccccccc} 
\\[-1.8ex]\hline 
\hline \\[-1.8ex] 
Statistic & \multicolumn{1}{c}{N} & \multicolumn{1}{c}{Mean} & \multicolumn{1}{c}{St. Dev.} & \multicolumn{1}{c}{Min} & \multicolumn{1}{c}{Pctl(25)} & \multicolumn{1}{c}{Pctl(75)} & \multicolumn{1}{c}{Max} \\ 
\hline \\[-1.8ex] 
pop, 2018 & 365 & 174,952.100 & 119,557.100 & 8,700 & 100,400 & 214,900 & 1,141,400 \\ 
job density, 2018 & 365 & 1.137 & 5.726 & 0.400 & 0.700 & 0.930 & 110.110 \\ 
distance to nearest met. area & 365 & 53.269 & 57.700 & 0.150 & 22.050 & 69.290 & 544.090 \\ 
distance to London & 365 & 201.558 & 173.634 & 0.150 & 76.180 & 278.880 & 1,003.950 \\ 
south of the UK & 365 & 0.463 & 0.499 & 0 & 0 & 1 & 1 \\ 
managerial jobs, 2020 & 363 & 12.009 & 4.013 & 3.600 & 9.000 & 14.300 & 27.900 \\ 
tech jobs, 2020 & 364 & 14.505 & 4.057 & 3.500 & 11.800 & 16.900 & 29.600 \\ 
skilled trade jobs, 2020 & 358 & 10.513 & 3.764 & 1.000 & 8.025 & 12.500 & 21.600 \\ 
professional jobs, 2020 & 364 & 21.223 & 6.902 & 4.400 & 16.775 & 24.850 & 71.600 \\ 
administrative jobs, 2020 & 359 & 9.965 & 2.738 & 3.200 & 8.100 & 11.400 & 21.300 \\ 
leisure jobs, 2020 & 362 & 9.261 & 2.827 & 2.800 & 7.300 & 11.400 & 17.800 \\ 
machine operation jobs, 2020 & 337 & 6.339 & 2.847 & 1.200 & 4.400 & 7.900 & 19.800 \\ 
earnings, 2019 & 360 & 592.184 & 81.129 & 437.600 & 534.625 & 633.875 & 893.200 \\ 
NVQ4+ & 365 & 39.329 & 11.076 & 15.000 & 31.800 & 45.300 & 100.000 \\ 
Virgin Media \%, 2020 & 365 & 0.152 & 0.141 & 0.000 & 0.018 & 0.241 & 0.753 \\ 
n. business est. per hab., 2019 & 365 & 0.057 & 0.164 & 0.023 & 0.038 & 0.056 & 3.174 \\ 
AM tests per hab., 2020 & 365 & 0.0005 & 0.0002 & 0.0001 & 0.0003 & 0.001 & 0.001 \\ 
\hline \\[-1.8ex] 
\end{tabular} 
\end{table}

\bibliographystyle{sageh}
\bibliography{bibliography}


\end{document}
