% interactcadsample.tex
% v1.03 - April 2017

\documentclass[]{interact}

\usepackage{epstopdf}% To incorporate .eps illustrations using PDFLaTeX, etc.
\usepackage{subfigure}% Support for small, `sub' figures and tables
%\usepackage[nolists,tablesfirst]{endfloat}% To `separate' figures and tables from text if required

\usepackage{natbib}% Citation support using natbib.sty
\bibpunct[, ]{(}{)}{;}{a}{}{,}% Citation support using natbib.sty
\renewcommand\bibfont{\fontsize{10}{12}\selectfont}% Bibliography support using natbib.sty

\theoremstyle{plain}% Theorem-like structures provided by amsthm.sty
\newtheorem{theorem}{Theorem}[section]
\newtheorem{lemma}[theorem]{Lemma}
\newtheorem{corollary}[theorem]{Corollary}
\newtheorem{proposition}[theorem]{Proposition}

\theoremstyle{definition}
\newtheorem{definition}[theorem]{Definition}
\newtheorem{example}[theorem]{Example}

\theoremstyle{remark}
\newtheorem{remark}{Remark}
\newtheorem{notation}{Notation}

% see https://stackoverflow.com/a/47122900


\usepackage{hyperref}
\usepackage[utf8]{inputenc}
\def\tightlist{}

\begin{document}

\articletype{ARTICLE TEMPLATE}

\title{WFH and broadband speed (title needs rework)}


\author{\name{A. N. Author$^{a}$, John Smith$^{b}$}
\affil{$^{a}$Taylor \& Francis, 4 Park Square, Milton Park, Abingdon, UK; $^{b}$Institut für Informatik, Albert-Ludwigs-Universität, Freiburg, Germany}
}

\thanks{CONTACT A. N. Author. Email: \href{mailto:latex.helpdesk@tandf.co.uk}{\nolinkurl{latex.helpdesk@tandf.co.uk}}, John Smith. Email: \href{mailto:john.smith@uni-freiburg.de}{\nolinkurl{john.smith@uni-freiburg.de}}}

\maketitle

\begin{abstract}
TBC
\end{abstract}

\begin{keywords}
covid; internet; working from home; broadband speed; time series
clusters
\end{keywords}

\hypertarget{introduction}{%
\section{Introduction}\label{introduction}}

\href{https://docs.google.com/document/d/1PWjkmgzWGYKR9wFogKYw7l-8mZLoORt593x-Tu-f2-M/edit\#heading=h.i5om1o8wpcd9}{our
Google doc}

\textbf{PARA1}: working from home is a challenge and maybe an
opportunity \citep[use arguments from][]{BUDNITZ2020102713} It has been
altered dramatically because of covid

\textbf{PARA2}: covind and working from home, cities, urban structure
Potentially useful readings for covid and cities:

\begin{itemize}
\tightlist
\item
  \url{https://www.coronavirusandtheeconomy.com/question/why-has-coronavirus-affected-cities-more-rural-areas}
\item
  \url{https://www.coronavirusandtheeconomy.com/question/will-coronavirus-cause-big-city-exodus}
\item
  \href{https://journals.sagepub.com/toc/EPB/current}{EPB commentaries}
\item
  \url{https://journals.sagepub.com/doi/full/10.1177/2399808320926912?casa_token=gh3SgKFCZ44AAAAA:MjjMWJ61DqlmxzkWQyy_wxPnU20QdyQYkC4fVFpHLPfLk7McwmQGkJ7x2Q7LZOjTf6vcYaqNwug}
\item
  \href{https://www.coronavirusandtheeconomy.com/question/what-has-coronavirus-taught-us-about-working-home}{WFM
  and covid, econ}
\item
  \href{https://www.coronavirusandtheeconomy.com/question/who-can-work-home-and-how-does-it-affect-their-productivity}{productivity
  and WFM}
\end{itemize}

It is important to become aware of working from home patterns because
of: - transport planning reasons (see para 1) - WFM might increase
coparing to pre-covid level

\textbf{PARA3}: How can we observe observe WFM? Directly through surveys
(\citet{felstead2020homeworking}). Add survey limitations (cost,
reprentativeness, time lag between survey and data availability) Data
gap. Passive data collection through internet speeds Another option
could be mobility data. How is internet speed data a better source than
mobility?

\textbf{PARA4}: Contention

\textbf{PARA5}: Data and methods

\textbf{PARA6}: Contribution:

\hypertarget{literature-review}{%
\section{Literature review}\label{literature-review}}

\hypertarget{broadband-studies-divides-broadband-tech-stuff}{%
\subsection{broadband studies, divides, broadband tech
stuff}\label{broadband-studies-divides-broadband-tech-stuff}}

\hypertarget{from-telecommuting-to-wfh}{%
\subsection{from telecommuting to
\#WFH}\label{from-telecommuting-to-wfh}}

Some new papers google recommended to me:

\begin{itemize}
\tightlist
\item
  \url{https://urbanstudies.uva.nl/binaries/content/assets/subsites/centre-for-urban-studies/working-paper-series/wps_43.pdf}
\item
  \url{https://link.springer.com/article/10.1007/s11116-020-10136-6}
\item
  \url{https://www.sciencedirect.com/science/article/pii/S0966692319311305}
\item
  check who cites the above and what they cite
\end{itemize}

\hypertarget{time-series-clustering}{%
\section{Time series clustering}\label{time-series-clustering}}

Desription of the method

\hypertarget{data-and-descriptive-statistics}{%
\section{Data and descriptive
statistics}\label{data-and-descriptive-statistics}}

Data details and some figures, descriptive stats

\hypertarget{results}{%
\section{Results}\label{results}}

Clusters, cluster description and aux regressions

\hypertarget{conclusions}{%
\section{Conclusions}\label{conclusions}}

\hypertarget{acknowledgements}{%
\section*{Acknowledgement(s)}\label{acknowledgements}}
\addcontentsline{toc}{section}{Acknowledgement(s)}

An unnumbered section,
e.g.~\texttt{\textbackslash{}section*\{Acknowledgements\}}, may be used
for thanks, etc.~if required and included \emph{in the non-anonymous
version} before any Notes or References.

\hypertarget{funding}{%
\section*{Funding}\label{funding}}
\addcontentsline{toc}{section}{Funding}

An unnumbered section,
e.g.~\texttt{\textbackslash{}section*\{Funding\}}, may be used for grant
details, etc.~if required and included \emph{in the non-anonymous
version} before any Notes or References.

\bibliographystyle{tfcad}
\bibliography{bibliography.bib}


\input{"appendix.tex"}


\end{document}
